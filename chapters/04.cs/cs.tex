\section{The Theory of Compressed Sensing} \label{cs}

Compressible signal is handled, how it works and why it works. So w

sampling theorem, s sparse signals need s samples which is far smaller than Nyquist Shannons Sampling Rate. Good part: sound theroy. Bad part: combinatorial in calculation. Worst case metrics, good average case metrics are still under research.

More in depth discussion about


\subsection{Modelling Signals}
$P(X)$ compressible signals like wavelets. It works if the sparse 

Idea of a dictionary, $X = D\alpha$
Dictionary, over complete representations. What is possible depends on the objective function

Finding a sparse space, can even be over complete and pretty much anything one wants.


\subsection{Objective Function for Compressed Sensing}
\begin{equation}\label{intro:eq:csclean}
\underset{X}{minimize} \: \left \| D_{dirty} - X \ast PSF \right \|_2^2 \: + \: \lambda \left \| P(X) \right \|_1
\end{equation}
L1 Relaxation of the Objective function
\begin{equation} \label{cs:eq:actual}
\underset{X}{minimize} \: \left \| D_{dirty} - x \ast PSF \right \|_2^2 + \lambda \: indicator(P(X))
\end{equation}

The actual CS formulation uses the indicator function for regularization. The indicator function is Zero for each element of $P(X)$ that is also zero, and 1 for each element that is not zero. This objective function \eqref{cs:eq:actual} is not Convex anymore. There are specialized optimization algorithms that minimize the objective \eqref{cs:eq:actual} like Matching Pursuit. 

Another solution is to use the L1 Relaxation instead of the indicator function. Convex Problem. And all the problems were solved.


Reconstruction in different spaces
\begin{equation}
\begin{split}
D_{dirty} &= F^{-1}MV \\
X &= D\alpha
\end{split}
\end{equation}

\begin{alignat*}{2}
	\underset{X}{minimize} \:& \left \| D_{dirty} - X \ast PSF \right \|_2^2 &&+  \lambda \left \| D^{-1}X \right \|_1 \\
	\underset{\alpha}{minimize} \:& \left \| D_{dirty} - D\alpha \ast PSF \right \|_2^2 &&+ \lambda \left \| \alpha \right \|_1 \\
	\underset{V_2}{minimize} \:& \left \| D_{dirty} - F^{-1} M V_2 \right \|_2^2 &&+ \lambda \left \| D^{-1}F^{-1}V_2\right \|_1 \\
	\underset{V_2}{minimize} \:& \left \| V - M V_2 \right \|_2^2 &&+ \lambda \left \| D^{-1}F^{-1}V_2\right \|_1
\end{alignat*}

All these equations should produce the same result in theory. In practice these different approaches have implications on the model and the chosen optimization algorithm. Not every $D^{-1}$ is well-defined. Coordinate descent works well on minimizing in the sparse space $\alpha$ directly. Also the convolution has to be explicitly handled when reconstructing in the image space.

L2 norm in the Image space weights more heavily on the lower frequency components.

\subsection{Optimization Algorithms}

Not all are equal. A good algorithm depends on what.

GuRoBi was used.


\subsection{Compressed Sensing Algorithms in Astronomy}

\subsubsection{SASIR}

\subsubsection{PURIFY}

\subsubsection{Vis-CS}



\subsection{Implementation In Casa}

Because of major and minor cycle
