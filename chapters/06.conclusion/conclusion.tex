\section{Future Compressed Sensing Reconstruction}
Compressed Sensing Reconstructions allow the exchange of priors. New prior knowledge can be incorporated without changing the underlying objective or the optimization algorithm.

Flexibility allows multiple ways to solve the same problem. Optimal solution does not exist yet. 

A proof of Compressed Sensing Reconstruction was implemented in CASA as a deconvolver. It was shown that with Compressed Sensing, one is potentially able to super-resolve sources. However, the current implementation has a quadratic memory requirement, it does not scale to image sizes which are used in practice. New interferometers like MeerKAT will require an even higher amount of pixels. The data amount will scale up by several factors compared to VLA and require scalable image reconstruction. Futhermore, they are built with wide field of view imaging in mind. 

Calibration for these instruments get more complicated. Compressed Sensing Reconstructions are flexible enough and can potentially improve self-calibration tasks.

The next step is to add the effects of wide field of view imaging and see how they scale.

