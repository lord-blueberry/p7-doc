\section{Sparsity and Compressed Sensing}

\begin{equation}\label{intro:underdetermined}
\begin{split}
Fx = y
\end{split}
\begin{split}
x &\in \mathcal{R}^n\\
y &\in \mathcal{C}^m\\
2m &< n
\end{split}
\end{equation}

\section{The Theory of Compressed Sensing} \label{cs}


If we have a bandwidth-limited Signal, how many samples do we need to reconstruct it? The Nyquist-Shannon Sampling Theory states if we have equidistant samples, our sampling rate needs to be more than twice the highest frequency component of the signal. If the sampling rate is lower, the reconstructed signals will be subject to aliasing, it will be different from the reality. However, Compressed Sensing is able to 

Compressible Signals, Sparsity


Compressible Signals, finding the true solution

\subsection{Compressible Signals and Priors}


\subsection{Compressed Sensing Algorithms in Radiointerferometry}

\subsection{Modelling}

\subsection{Implementation In Casa}

\subsubsection{The Major and Minor Cycle}
